
\usepackage{ctex}
%%\usepackage{ctexcap} % 需要将 ctexcap.sty 里与 ctex 宏包重复的部分注释,即不用加载相同包
\usepackage{relsize}                 % 调整公式字体大小:\mathsmaller, \mathlarger
%\usepackage{times}
\usepackage{fontspec,xunicode,xltxtra} % XeLaTeX相关字体字库

%%%%%%%%%%%%%%%%%%%%%%%%%%%%%%%%%%%%%%%%%%%%%%%%%%%


\usepackage{etex}  % 解决宏包 no room for 。。。的错误
%%%%%%%%%%% 版本修改记录宏包 %%%%%%%%%%%%%%%%%%%%%%
%\usepackage[nochapter]{vhistory}

\usepackage{caption2}                               % 按标准, 去掉图表号后面的:
\usepackage{lipsum}   % To generate test text 产生测试文本

%%%%%%%%%%%%%% 颜色 %%%%%%%%%%%%%%%%%%%%%%%%%%%%%%%%%%%%%%
\usepackage[table,dvipsnames,svgnames]{xcolor}
\usepackage{xxcolor}


%%%%%%%%%%%% 合并PDF文档 与tikz extern 冲突%%%%%%%%%%%%%%%%%%%%%%
%\usepackage{pdfpages}

%%%%%%%%%%%% 图表标题格式包 %%%%%%%%%%%%%%%%%%%%%%
\usepackage[Euler]{upgreek}
\usepackage{amsmath,amsfonts,amssymb} %
\usepackage{latexsym,bm}        %公式符号
\usepackage[misc,electronic,clock]{ifsym} %电气符号
\usepackage{dingbat}
\usepackage[Omega,upmu]{gensymb}
\usepackage{wasysym}
\usepackage{marvosym}

%%%%%%%%%%%%%%%%    插图  %%%%%%%%%%%%%%%%%%%%%%%%%%%%%%%%%%%%
\usepackage{graphicx}       %插图宏包
\usepackage{wallpaper}    %绘图文绕排宏包,页面背景宏包,
\usepackage{picinpar} %与pgf宏包冲突

%%%%%%%%%%%%%% 彩色表格,表格线条 %%%%%%%%%%%%%%%%%%%%%%%%%%%%%%%%%%%%
\usepackage{booktabs,slashbox,colortbl,longtable,multirow,tabularx,dcolumn}                         %表格粗线,斜线,彩色表格,长表格
%%%%%%%%%%%%% 页版面,边距设置 %%%%%%%%%%%%%%%%%%%%%%%%%%%%%%%%%%
\usepackage[top=2.54cm,bottom=2.54cm,left=2.15cm,right=2.5cm,includehead,includefoot]{geometry}
%上下2.54,左右2
%%%%%%%%%%% 中文书签中文复制 %%%%%%%%%%%%%%%%%%%%%%%%%%%%%%%%%%
\usepackage[colorlinks=no,
            citecolor=blue,
            linkcolor=blue,
            anchorcolor=green,
            urlcolor=blue,
%% 与attachfile2冲突           pdfauthor={wangfan},%作者
%%            pdfkeywords={latex},%关键词
%%            pdfsubject={latex},%主题
%%            pdftitle={handbook of latex},%标题
            CJKbookmarks=true,
            pdfborder={0 0 0},
            bookmarksnumbered=true,
            bookmarksopen=false,
            xetex,
            ]{hyperref}
%\usepackage{ccmap}               % 使生成的PDF文件支持复制等,对pdflatex
%
%
%%%%%%%%%%%%%%%%%%%%%%%%%%%%%%%%%%%%%%%%%%%%%%%%%%%%%%%%%%%%%%%%%%%
\usepackage{titletoc}           %目录格式包


%%%%%%%%%%%%%%%%%%%%%%%%%%%%%%%%%%%%%%%%%%%%%%%%%%%%%%%%%%%%%%%%%%%标题中文化
\usepackage[bf,small,raggedright,indentafter,pagestyles]{titlesec}
        %其中bf设置章节标题的字体为黑体,这也是默认值,可以略去。
        %此外,还可以设 为rm(罗马体), sf(无衬线体), tt(打字机体), md(中等黑度),
        %up(直立体), it(意大利斜体), sl(机械斜体), sc(小体大写字母)。
        %small设置标题字体的尺寸,还可设为big(默认), medium, tiny。
        %center使标题居中,还可以设为raggedleft(居左,默认), raggedright(居右)。
        %indentafter相当于宏包indentfirst的作用,使标题下面的第一个段落正常缩进。
        %pagestyles是申明后面要自定义页面样式。

%%%%%%%%%%%%%%%%%%%%%%%%%%%%\usepackage{tocloft}

\usepackage{fancyhdr}       %自定义页眉页脚

%%%%%%%%%%%% 抄录环境 %%%%%%%%%%%%%
\usepackage{fancyvrb,sverb}
%

%%%%%%%%%%%% 盒子环境 %%%%%%%%%%%%%
\usepackage{framed}


%%%%%%%%%%%%% ASY绘图宏包 %%%%%%%%%%%%%%%%%%%%%%
\usepackage{asymptote}

%%%%%%%%%%%%% SHAPE宏包 %%%%%%%%%%%%%%%%%%%%%%
\usepackage{shapepar}

%%%%%%%%%%%%% 图片放置宏包 不放在文字前面 %%%%%%%%%%%%%%%%%%%%%%
\usepackage{flafter,float}

%%%%%%%%%%%%% 下划线宏包 %%%%%%%%%%%%%%%%%%%%%%
 \usepackage[normalem]{ulem}%`加入宏包`
 \usepackage{CJKfntef} %汉字下划线宏包

%%%%%%%%%%%%%% 页码宏包 (与动画宏包冲突)%%%%%%%%%%%%%%%%%%%%%%
\usepackage{lastpage}

%%%%%%%%%%%%% 动画宏包 %%%%%%%%%%%%%%%%%%%%%%
\usepackage{animate} % 与 tikz 部分宏包冲突

%%%%%%%%%%%% 行号宏包 %%%%%%%%%%%%%%%%%%%%%%
\usepackage[left]{lineno} %与 tikz 宏包冲突

%%%%%%%%%%%%% 视频宏包 %%%%%%%%%%%%%%%%%%%%%%
\usepackage{movie15} % 与 tikz 部分宏包冲突

%
%%%%%%%%%%%%% 时间宏包 %%%%%%%%%%%%%%%%%%%%%%
%\usepackage{tdclock}


%%%%%%%%%%%%% 短列表宏包 %%%%%%%%%%%%%%%%%%%%%%
%\usepackage{shortlst}
%
%%%%%%%%%%%%% 列表编号宏包 %%%%%%%%%%%%%%%%%%%%%%
\usepackage{enumerate}



%%%%%%%%%%%% 脚注尾注宏包 %%%%%%%%%%%%%%%%%%%%%%
\usepackage{threeparttable,endnotes}

%%%%%%%%%%%% 索引表 %%%%%%%%%%%%%%%
\usepackage{makeidx}\makeindex

%%%%%%%%%%% 索引宏包 %%%%%%%%%%%%%%%%%%%%%%
%\usepackage{xesearch}
%\usepackage{xeindex}\makeindex


%%%%%%%%%%%% 引用包 %%%%%%%%%%%%%%%
\usepackage{cite}  %实现[1-4]方式引用多个参考文献

%%%%%%%%%%%% 双栏排版宏包 %%%%%%%%%%%%%%%
\usepackage{flushend,cuted}
\usepackage{multicol} %多栏排版
%
%\usepackage{fancybox} %与framed宏包冲突
%%%%%%%%%%%%%% 生成HTML宏包 %%%%%%%%%%%%%%%%%%%%%%
%%\usepackage{html,epsf}


%%%%%%%%%%% 附件宏包 %%%%%%%%%%%%%%%%%%%%%%
\usepackage{attachfile2}

%%%%%%%%%%% 目录结构图宏包 %%%%%%%%%%%%%%%%%%%%%%
\usepackage{dirtree}

%%%%%%%%%%% 柱状图宏包 %%%%%%%%%%%%%%%%%%%%%%
\usepackage{bardiag}


%%%%%%%%%%% 书签宏包 %%%%%%%%%%%%%%%%%%%%%%
\usepackage[open,openlevel=0,atend]{bookmark}
