\bookmarksetup{bold,color=grass}
\chapter{附\qquad 录}
\bookmarksetup{bold=false,color={}}
\thispagestyle{fancy}   \fancyhead[R]{\song\wuhao 附录}

\section{常见参数}

\subsection{位置参数}
\begin{lstlisting}[language={[LaTeX]TeX}]
\parbox`[位置参数]`{`长度`}{`内容`}
\begin{figure(table)}[`位置参数`]
\end{lstlisting}

\fcolorbox{white}{lightgray}{parbox 位置参数有 t b s c 四种}

\begin{tabular}{ll}

  % after \\: \hline or \cline{col1-col2} \cline{col3-col4} ...
  t & top 文本放置盒子顶部 \\
  b & bottom 文本放置盒子底部 \\
  s & spread 伸展行间距充满盒子 \\
  c & center 盒子中间 \textcolor[rgb]{0.00,0.00,1.00}{默认属性} \\

\end{tabular}

\fcolorbox{white}{lightgray}{\parbox{12cm}{figure table 位置参数有 h
t b p 四种,可同时写 4 种在括号内,确定优先顺序。}}

\begin{tabular}{ll}

  % after \\: \hline or \cline{col1-col2} \cline{col3-col4} ...
  h & here 固定位置 \\
  t & top 页顶 \\
  b & bottom 页尾 \\
  p & floatpage 单独的浮动页 \\
\end{tabular}

\subsection{表格参数}
\fcolorbox{white}{lightgray}{tabularx 环境中可自定义表格宽度}
\begin{lstlisting}[language={[LaTeX]TeX}]
\begin{tabularx}{14cm}{lrcxp{2cm}}
%`分别是左右中等各种对齐方式,14cm为总长度`
\end{lstlisting}
\begin{tabular}{ll}

  % after \\: \hline or \cline{col1-col2} \cline{col3-col4} ...

  l & left 左对齐\\
  c & center 中对齐 \\
  r & right 右对齐 \\
  x & 根据总长度自动换行\\
  p\{宽度\} & 指定表格宽度,超出可自动换行 \\

\end{tabular}

\subsection{居中环境}

\rowcolors{1}{lightgray}{}
\begin{tabularx}{14cm}{lp{4cm}X}
  行居中 & \verb|\centerline{}| & 将一行文本居中并与上下文空出一行行距,常用于表格,图形 \\
  所有对象居中 &  \verb|{\centering object}| & 将 centering 之后的对象全部居中\\
  居中环境 & \verb|\begin{center}|\verb|\end{center}| &如在换行后加大行距,可使用\verb|\\[4mm]|换行符后的距离为可选参数。\\
\end{tabularx}

\section{$*$的区别}

\begin{table}[H]
  \centering
  \caption{$*$的用法区别}\label{star_use}
\rowcolors{1}{lightgray}{}
\begin{tabularx}{14cm}{lp{6cm}X}
  \toprule
  文档属性 & \verb|\begin{CJK*}{}{}| & 带$*$会自动忽略汉字后的空格及换行,用 \~{} 来加入空隙\\
抄录环境 & \verb|\begin{verbatim*}|、\verb|\verb*| &  带$*$ 会将空格以\verb*| |显示 \\
公式环境 & \verb|\begin{equation*}| & 带$*$的不参加自动编号 \\
表格环境 &\verb|\begin{longtable*}| & 表格计数器不加 1\\
&\verb|\begin{tabluar*}| & 增加宽度参数,同 tabluarx ,但不能使用脚注\\
系统命令 & \verb|\newcommand*|、\verb|\renewcommand*| & 带星号后命令不能含换行等参数\\
超链接 & \verb|\ref*| & 带星号会注释掉超链接\\
长度填充 & \verb|\hspace*{} \vspace*{}| & hspace 带星号的若在一行开始或结尾则系统删除空白,vspace 带星号若在新一页的开始或结尾则删除此空白\\
缩放对象 & \verb|\resizebox*{}{}{}| & 不带星号第 2 个括号为高度,带星号为总高度\\
  \bottomrule
\end{tabularx}
\end{table}



\section{listing 宏包可高亮关键字的程序语言}
\includepdfmerge{body/language_list}

\section{常用符号}
\begin{description}
  \item[省略号] cdots  ldots : $\cdots  \ldots$
  \item[波浪号] nbs: \~{}
  \item[引号] \`{},$'"$:` ' ``\,"
\end{description}

三个上下标后面都带括号。\\
\begin{table}[!h]
  \centering
  \caption{特殊字符}\label{sym}
  \begin{tabular}{ccccccccc}
    \toprule
    % after \\: \hline or \cline{col1-col2} \cline{col3-col4} ...
    \# & \$ & \% & \{ & \} & \~{} & \^{} &\_ & $\backslash$ \\
    $\backslash$\# & $\backslash$ \$ & $\backslash$ \% & $\backslash$\{ & $\backslash$\{
    & $\backslash$\~{}\{\}  &  $\backslash$\^{}\{\} &  $\backslash$\_ & \$$\backslash$backslash\$ \\
    \bottomrule
  \end{tabular}
\end{table}



\section{beamer 常用设置命令}

\subsection{设置样式、颜色、字体}

\begin{lstlisting}
`样式设置`
\setbeamertemplate{beamer`元素`}{`定义`}
\setbeamertemplate{beamer`元素`}[`参数`]

`颜色设置`
\setbeamercolor{beamer`元素`}{fg=`字体颜色`,bg=`背景颜色`}

`字体设置`
\setbeamerfont{beamer `元素`}{`定义`}
`定义有:`
size=	 \small,\large `等字体尺寸`
series=	 `默认`\mdseries,`可选`\mdseries
shape=	 `默认`\upshape,`可选`\itshape,\scshape
family=	 `默认`\sffamily,`可选`\rmfamily,\tffamily
\end{lstlisting}

\clearpage
