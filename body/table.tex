\section{表格设计}


\begin{lstlisting}[language={[LaTeX]TeX}]
\thetable `表名:原为 table 1,后改为汉字“ 表 1.1 ”` \tablename
`表名内容:对应 caption 后的内容`
\end{lstlisting}
表格在 latex
中的应用主要是三线表格,斜线表格,长表格,合并表格,彩色表格,固定列宽自适应表格。分别要用到以下宏包。

\subsection{表格概述}


\fcolorbox{white}{lightyellow} {
\parbox{12cm}{
\begin{enumerate}
  \item 三线命令:toprule midrule bottomrule cmidrule\{m-n\}
  \item 斜线命令:backslashbox\{\}  slashbox\{\}
  \item 彩色命令:columncolor rowcolor cellcolor
  \item 自适应宽度命令:P\{长度\} X
  \item 合并行列命令:multicolumn\{2\}\{c\}\{内容\} multirow\{3\}\{*\}\{$\backslash$centering 爽\}
  \item 长表格命令:endfirsthead endhead endfoot endlastfoot $\backslash$hline$\backslash$hline 表示加粗
\end{enumerate}
} }


\subsection{三线表格 booktab宏包}
\index{命令!\verb$\begin{table}$}
\index{宏包!booktab}

\begin{lstlisting}
\begin{table}[ht]
  \centering
  \caption{表格类型}\label{table_style}
  \begin{tabular}{lll}
    \toprule
    % after \\: \hline or \cline{col1-col2} \cline{col3-col4} ...
    表格类型 & 所用宏包 & 对应命令\\
    \midrule
    三线表格 & booktabs  & toprule,midrule,bottomrule \\
    斜线表格 & slashbox \\
    彩色表格 & colortbl \\
    自适应表格 & tabularx\\
    合并行列表格 & multirow \\
    长表格& longtable \\
    \bottomrule
  \end{tabular}
\end{table}
\end{lstlisting}
\begin{table}[ht]
  \centering
  \caption{表格类型}\label{table_style}
  \begin{tabular}{lll}
    \toprule
    % after \\: \hline or \cline{col1-col2} \cline{col3-col4} ...
    表格类型 & 所用宏包 & 对应命令\\
    \midrule
    三线表格 & booktabs  & toprule,midrule,bottomrule \\
    斜线表格 & slashbox \\
    彩色表格 & colortbl \\
    自适应表格 & tabularx\\
    合并行列表格 & multirow \\
    长表格& longtable \\
    \bottomrule
  \end{tabular}
\end{table}

%%%%%%%%%%%%%%%%%%%%%%%%%%%%%%%%%%%%%%%%%%%%%%%%%%%%%%%%%%%%

\subsection{跨行表格 multirow宏包 }
\index{宏包!multirow}
\index{命令!\verb$\multirow$}

\begin{lstlisting}[language={[LaTeX]TeX}]
 Windows & MikTeX & TeXnicCenter &
 \multirow{3}{*}{\centering  `爽`}\\
\end{lstlisting}

\begin{latexcmd}[label= 跨行表格命令]
\multirow{所跨行数}[补偿]{数据宽度}[位移量]{数据}
数据宽度用 * 号表示数据的自然宽度,可不加花括号
位移量正表示上移,负表示下移
\end{latexcmd}

\begin{table}[htbp]
\caption{跨行表格} \centering
\begin{tabular}{lllc}
\toprule

操作系统 & 发行版 & 编辑器 & 用户体验\\

\midrule Windows & MikTeX & TeXnicCenter&
\multirow{3}{*}{\centering 爽}\\

Unix/Linux&TeXLive&Emacs\\

MacOS&MacTeX&TeXShop\\

\bottomrule
\end{tabular}
\end{table}


\subsection{跨列表格 cmidrule multicolumn }

\index{命令!\verb$\toprule$}
\index{命令!\verb$\midrule$}
\index{命令!\verb$\bottomrule$}
\index{命令!\verb$\cmidrule$}
\index{命令!\verb$\multicolumn$}

cmidrule 为 booktabs 宏包用来划跨列横线 multicolumn 用于跨列内容,也可用来设置不同行不同的对齐方式。


\begin{lstlisting}[language={[LaTeX]TeX}]
&\multicolumn{2}{c}{`常用工具`}\\
\end{lstlisting}

\begin{lstlisting}[language={[LaTeX]TeX}]
\toprule
 & \multicolumn{2}{c}{`常用工具`}\\
\cmidrule{2-3}
`操作系统` & `发行版` & `编辑器` \\
\end{lstlisting}


\begin{table}[htbp]
\caption{跨行列表格} \centering
\begin{tabular}{lll}
\toprule
&\multicolumn{2}{c}{常用工具}\\
\cmidrule{2-3}
操作系统 & 发行版 & 编辑器 \\
\midrule
Windows &  MikTeX & TeXnicCenter\\
Unix/Linux & TeXLive & Emacs\\
MacOS & MacTeX & TeXShop\\
\bottomrule
\end{tabular}
\end{table}

\begin{cmd}
\begin{table}[htbp]
\caption{跨行列表格} \centering
\begin{tabular}{lll}
\toprule
&\multicolumn{2}{c}{常用工具}\\
\cmidrule{2-3}
操作系统 & 发行版 & 编辑器 \\
\midrule
Windows &  MikTeX & TeXnicCenter\\
Unix/Linux & TeXLive & Emacs\\
MacOS & MacTeX & TeXShop\\
\bottomrule
\end{tabular}
\end{table}
\end{cmd}

%%%%%%%%%%%%%%%%%%%%%%%%%%%%%%%%%%%%%%%%%%%%%%%%%%%%%%%%%%%%

\subsection{斜线表格 diagbox}
\index{宏包!diagbox}
\index{命令!\verb$\backslashbox$}

原先 slashbox 宏包被 diagbox 宏包替代,并且 diagbox 可实现更高级的功能。
\begin{lstlisting}[language={[LaTeX]TeX}]
\backslashbox{`内容1`}{`内容2`} %`1在左,2在右`
\slashbox{`内容1`}{`内容2`}
\end{lstlisting}

\begin{table}[ht]
  \centering
  \caption{表格斜线}\label{slant}
  \begin{tabular}{|l|l|}
    \hline
    % after \\: \hline or \cline{col1-col2} \cline{col3-col4} ...
    \backslashbox{内容1}{内容2} & \slashbox{内容1}{内容2} \\
    \hline
     &    \\
    \hline
  \end{tabular}
\end{table}

\begin{cmd}
  \begin{table}[ht]
  \centering
  \caption{表格斜线}\label{slant}
  \begin{tabular}{|l|l|}
    \hline
    % after \\: \hline or \cline{col1-col2} \cline{col3-col4} ...
    \backslashbox{内容1}{内容2} & \slashbox{内容1}{内容2} \\
    \hline
     &    \\
    \hline
  \end{tabular}
\end{table}
\end{cmd}

%%%%%%%%%%%%%%%%%%%%%%%%%%%%%%%%%%%%%%%%%%%%%%%%%%%%%%%%%%%%

\subsection{彩色表格 colortbl}

\index{命令!\verb$\rowcolor$}
\index{命令!\verb$\columncolor$}
\index{命令!\verb$\cellcolor$}
\index{宏包!colortbl}

彩色表格需要使用colortbl宏包提供的一些命令:$\backslash$columncolor、
$\backslash$rowcolor、$\backslash$cellcolor等。

\begin{table}[htbp]
\caption{彩色表格} \centering
%%%%%%%%%%%%%% [0cm][0cm]表明向左右各扩展多少距离
\begin{tabular}{>{\columncolor{lightgray}[0cm][0cm]}lll}
\toprule
操作系统 & 发行版 &\cellcolor{lightgray} 编辑器 \\
\midrule
Windows&MikTeX&TeXnicCenter\\
\rowcolor{lightgray}
Unix/Linux&TeXLive & Emacs\\
MacOS&MacTeX&TeXShop\\
\bottomrule
\end{tabular}
\end{table}

\begin{cmd}
  \begin{table}[htbp]
\caption{彩色表格} \centering
%%%%%%%%%%%%%% [0cm][0cm]表明向左右各扩展多少距离
\begin{tabular}{>{\columncolor{lightgray}[0cm][0cm]}lll}
\toprule
操作系统 & 发行版 &\cellcolor{lightgray} 编辑器 \\
\midrule
Windows&MikTeX&TeXnicCenter\\
\rowcolor{lightgray}
Unix/Linux&TeXLive & Emacs\\
MacOS&MacTeX&TeXShop\\
\bottomrule
\end{tabular}
\end{table}
\end{cmd}


%%%
\index{命令!\verb$\rowcolors$}

下面这行代码用于设置不同的行颜色不同,命令为可作用于每一行的命令,如$\backslash$hline 经常忽略,
行号为从第几行开始着色,一般从第一开始,颜色选项为空则不加颜色。这行代码必须加在表格环境 table,由 xcolor 宏包配合使用。\\
\color{grass}
\verb(\rowcolors[命令]{行号}{奇数行颜色}{偶数行颜色}(\\
\color{black}

需要在添加 xcolor 宏包时加入 table 选项:
\begin{Verbatim}[formatcom=\color{grass},frame=single]
\usepackage[table]{xcolor}%加入宏包
\rowcolors[\hline]{1}{lightblue}{whiteblue}

代码样例:
\begin{table}[htbp]
\caption{轮换颜色表格} \centering
\rowcolors[\hline]{1}{lightblue}{whiteblue}

\begin{tabular}{lll}
操作系统 & 发行版 & 编辑器 \\
Windows&MikTeX&TeXnicCenter\\
Unix/Linux&TeXLive & Emacs\\
MacOS&MacTeX&TeXShop\\
\end{tabular}
\end{table}
\end{Verbatim}


\begin{table}[htbp]
\caption{轮换颜色表格} \centering
\rowcolors[\hline]{1}{lightblue}{whiteblue}
%%%%%%%%%%%%%% [0cm][0cm]表明向左右各扩展多少距离
\begin{tabular}{lll}
操作系统 & 发行版 & 编辑器 \\
Windows&MikTeX&TeXnicCenter\\
Unix/Linux&TeXLive & Emacs\\
MacOS&MacTeX&TeXShop\\
\end{tabular}
\end{table}



%%%%%%%%%%%%%%%%%%%%%%%%%%%%%%%%%%%%%%%%%%%%%%%%%%%%%%%%%%%%

\subsection{自定义宽度表格 tabularx }

\index{宏包!tabularx}

若想控制整个表格的宽度可以使用 tabularx 宏包,X
参数表示某栏可以自动折行。也可以用p\{长度\}来手动折行\\
\begin{Verbatim}
\begin{table}[htbp]
\caption{控制表格宽度} \centering
\begin{tabularx}{350pt}{lXlX}
\toprule 李白 &
平林漠漠烟如织,寒山一带伤心碧。暝色入高楼,有人楼上愁。
玉梯空伫立,宿鸟归飞急。何处是归程,长亭更短亭。& 泰戈尔 &
夏天的飞鸟,飞到我的窗前唱歌,又飞去了。秋天的黄叶,它
们没有什么可唱,只叹息一声,飞落在那里。\\
\bottomrule
\end{tabularx}
\end{table}
\end{Verbatim}

\begin{table}[htbp]
\caption{控制表格宽度} \centering
%%%%%%%%%%%%%% x表示这一列根据总长度自动折行 %%%%%%%%%%%%%%%%%%%%
\begin{tabularx}{350pt}{lXlX}
\toprule 李白 &
平林漠漠烟如织,寒山一带伤心碧。暝色入高楼,有人楼上愁。
玉梯空伫立,宿鸟归飞急。何处是归程,长亭更短亭。& 泰戈尔 &
夏天的飞鸟,飞到我的窗前唱歌,又飞去了。秋天的黄叶,它
们没有什么可唱,只叹息一声,飞落在那里。\\
\bottomrule
\end{tabularx}
\end{table}

\subsection{长表格}


\index{宏包!longtable}
长表格 longtable 宏包,主要设计表头,续页表头,表尾,未页表尾这些部分。\\
\color{grass}

参考代码如下:

\begin{latexcmd}[label=\LaTeX 长表格代码]
\rowcolors{1}{lightgray}{}
\begin{longtable}[H]{p{3cm}p{8cm}c}
\caption{标题1} \label{strap_sb710} \\
\toprule \multicolumn{1}{c}{\textbf{信号名称 }} &  %居中
\multicolumn{1}{l}{\textbf{说明}} & %居左
\multicolumn{1}{c}{\textbf{备注 }} \\ \midrule %居中
\endfirsthead

\multicolumn{3}{c}%
{{\kai \thetable{}标题1 -~- 接上页}} \\
\toprule \multicolumn{1}{c}{\textbf{信号名称}} &  %居中
\multicolumn{1}{l}{\textbf{说明 }} & %居左
\multicolumn{1}{c}{\textbf{备注}} \\ \midrule  %居中
\endhead

 \multicolumn{3}{r}{{\kai 接下页}} \\ \bottomrule
\endfoot
\bottomrule
\endlastfoot

\end{longtable}

\end{latexcmd}

\begin{shaded}
\begin{Verbatim}
\begin{longtable}{|l|l|l|}
\caption{长表格} \label{grid_mlmmh} \\

\hline \multicolumn{1}{|c|}{\textbf{表头 1 }} &
\multicolumn{1}{c|}{\textbf{表头 2 }} &
\multicolumn{1}{c|}{\textbf{表头 3 }} \\ \hline
\endfirsthead

\multicolumn{3}{c}%
{{\kai  \thetable{} -~- 接上页}} \\
\hline \multicolumn{1}{|c|}{\textbf{表头 1 }} &
\multicolumn{1}{c|}{\textbf{表头 2 }} &
\multicolumn{1}{c|}{\textbf{表头 3 }} \\ \hline
\endhead

\hline  \multicolumn{3}{|r|}{{\kai 接下页}} \\ \hline
\endfoot

\hline \hline
\endlastfoot


0 & (1, 11, 13725) &...\\%表内容
......
\end{longtable}
\end{Verbatim}
\end{shaded}
\normalcolor

\begin{center}
\begin{longtable}{|l|l|l|}
\caption{长表格} \label{grid_mlmmh} \\

\hline \multicolumn{1}{|c|}{\textbf{表头 1 }} &
\multicolumn{1}{c|}{\textbf{表头 2 }} &
\multicolumn{1}{c|}{\textbf{表头 3 }} \\ \hline
\endfirsthead

\multicolumn{3}{c}%
{{\kai  \thetable{} -~- 接上页}} \\
\hline \multicolumn{1}{|c|}{\textbf{表头 1 }} &
\multicolumn{1}{c|}{\textbf{表头 2 }} &
\multicolumn{1}{c|}{\textbf{表头 3 }} \\ \hline
\endhead

\hline  \multicolumn{3}{|r|}{{\kai 接下页}} \\ \hline
\endfoot

\hline \hline
\endlastfoot

0 & (1, 11, 13725) & (1, 12, 10980), (1, 13, 8235), (2, 2, 0), (3, 1, 0) \\
2745 & (1, 12, 10980) & (1, 13, 8235), (2, 2, 0), (2, 3, 0), (3, 1, 0) \\
5490 & (1, 12, 13725) & (2, 2, 2745), (2, 3, 0), (3, 1, 0) \\
8235 & (1, 12, 16470) & (1, 13, 13725), (2, 2, 2745), (2, 3, 0), (3, 1, 0) \\
10980 & (1, 12, 16470) & (1, 13, 13725), (2, 2, 2745), (2, 3, 0), (3, 1, 0) \\
13725 & (1, 12, 16470) & (1, 13, 13725), (2, 2, 2745), (2, 3, 0), (3, 1, 0) \\
16470 & (1, 13, 16470) & (2, 2, 2745), (2, 3, 0), (3, 1, 0) \\
19215 & (1, 12, 16470) & (1, 13, 13725), (2, 2, 2745), (2, 3, 0), (3, 1, 0) \\
21960 & (1, 12, 16470) & (1, 13, 13725), (2, 2, 2745), (2, 3, 0), (3, 1, 0) \\
24705 & (1, 12, 16470) & (1, 13, 13725), (2, 2, 2745), (2, 3, 0), (3, 1, 0) \\
27450 & (1, 12, 16470) & (1, 13, 13725), (2, 2, 2745), (2, 3, 0), (3, 1, 0) \\
30195 & (2, 2, 2745) & (2, 3, 0), (3, 1, 0) \\
32940 & (1, 13, 16470) & (2, 2, 2745), (2, 3, 0), (3, 1, 0) \\
35685 & (1, 13, 13725) & (2, 2, 2745), (2, 3, 0), (3, 1, 0) \\
38430 & (1, 13, 10980) & (2, 2, 2745), (2, 3, 0), (3, 1, 0) \\
41175 & (1, 12, 13725) & (1, 13, 10980), (2, 2, 2745), (2, 3, 0), (3, 1, 0) \\
43920 & (1, 13, 10980) & (2, 2, 2745), (2, 3, 0), (3, 1, 0) \\
46665 & (2, 2, 2745) & (2, 3, 0), (3, 1, 0) \\
49410 & (2, 2, 2745) & (2, 3, 0), (3, 1, 0) \\
52155 & (1, 12, 16470) & (1, 13, 13725), (2, 2, 2745), (2, 3, 0), (3, 1, 0) \\
54900 & (1, 13, 13725) & (2, 2, 2745), (2, 3, 0), (3, 1, 0) \\
57645 & (1, 13, 13725) & (2, 2, 2745), (2, 3, 0), (3, 1, 0) \\
60390 & (1, 12, 13725) & (2, 2, 2745), (2, 3, 0), (3, 1, 0) \\
63135 & (1, 13, 16470) & (2, 2, 2745), (2, 3, 0), (3, 1, 0) \\
65880 & (1, 13, 16470) & (2, 2, 2745), (2, 3, 0), (3, 1, 0) \\
68625 & (2, 2, 2745) & (2, 3, 0), (3, 1, 0) \\
71370 & (1, 13, 13725) & (2, 2, 2745), (2, 3, 0), (3, 1, 0) \\
74115 & (1, 12, 13725) & (2, 2, 2745), (2, 3, 0), (3, 1, 0) \\
76860 & (1, 13, 13725) & (2, 2, 2745), (2, 3, 0), (3, 1, 0) \\
79605 & (1, 13, 13725) & (2, 2, 2745), (2, 3, 0), (3, 1, 0) \\
82350 & (1, 12, 13725) & (2, 2, 2745), (2, 3, 0), (3, 1, 0) \\
85095 & (1, 12, 13725) & (1, 13, 10980), (2, 2, 2745), (2, 3, 0), (3, 1, 0) \\
87840 & (1, 13, 16470) & (2, 2, 2745), (2, 3, 0), (3, 1, 0) \\
90585 & (1, 13, 16470) & (2, 2, 2745), (2, 3, 0), (3, 1, 0) \\
93330 & (1, 13, 13725) & (2, 2, 2745), (2, 3, 0), (3, 1, 0) \\
96075 & (1, 13, 16470) & (2, 2, 2745), (2, 3, 0), (3, 1, 0) \\
98820 & (1, 13, 16470) & (2, 2, 2745), (2, 3, 0), (3, 1, 0) \\
101565 & (1, 13, 13725) & (2, 2, 2745), (2, 3, 0), (3, 1, 0) \\
104310 & (1, 13, 16470) & (2, 2, 2745), (2, 3, 0), (3, 1, 0) \\
107055 & (1, 13, 13725) & (2, 2, 2745), (2, 3, 0), (3, 1, 0) \\
109800 & (1, 13, 13725) & (2, 2, 2745), (2, 3, 0), (3, 1, 0) \\
112545 & (1, 12, 16470) & (1, 13, 13725), (2, 2, 2745), (2, 3, 0), (3, 1, 0) \\
115290 & (1, 13, 16470) & (2, 2, 2745), (2, 3, 0), (3, 1, 0) \\
118035 & (1, 13, 13725) & (2, 2, 2745), (2, 3, 0), (3, 1, 0) \\
120780 & (1, 13, 16470) & (2, 2, 2745), (2, 3, 0), (3, 1, 0) \\
123525 & (1, 13, 13725) & (2, 2, 2745), (2, 3, 0), (3, 1, 0) \\
126270 & (1, 12, 16470) & (1, 13, 13725), (2, 2, 2745), (2, 3, 0), (3, 1, 0) \\
129015 & (2, 2, 2745) & (2, 3, 0), (3, 1, 0) \\
131760 & (2, 2, 2745) & (2, 3, 0), (3, 1, 0) \\
134505 & (1, 13, 16470) & (2, 2, 2745), (2, 3, 0), (3, 1, 0) \\
137250 & (1, 13, 13725) & (2, 2, 2745), (2, 3, 0), (3, 1, 0) \\
139995 & (2, 2, 2745) & (2, 3, 0), (3, 1, 0) \\
142740 & (2, 2, 2745) & (2, 3, 0), (3, 1, 0) \\
145485 & (1, 12, 16470) & (1, 13, 13725), (2, 2, 2745), (2, 3, 0), (3, 1, 0) \\
148230 & (2, 2, 2745) & (2, 3, 0), (3, 1, 0) \\
150975 & (1, 13, 16470) & (2, 2, 2745), (2, 3, 0), (3, 1, 0) \\
153720 & (1, 12, 13725) & (2, 2, 2745), (2, 3, 0), (3, 1, 0) \\
156465 & (1, 13, 13725) & (2, 2, 2745), (2, 3, 0), (3, 1, 0) \\
159210 & (1, 13, 13725) & (2, 2, 2745), (2, 3, 0), (3, 1, 0) \\
161955 & (1, 13, 16470) & (2, 2, 2745), (2, 3, 0), (3, 1, 0) \\
164700 & (1, 13, 13725) & (2, 2, 2745), (2, 3, 0), (3, 1, 0) \\
\end{longtable}
\end{center}

如果续页不用表头,也可以直接将原 table 环境改成 longtable 环境即可
\begin{longtable}[htbp]{|l|l|l|}
\caption{长表格2} \label{grid_mlmmh} \\
\hline
表头 1 & 表头 2 & 表头 3\\
\hline
115290 & (1, 13, 16470) & (2, 2, 2745), (2, 3, 0), (3, 1, 0) \\
118035 & (1, 13, 13725) & (2, 2, 2745), (2, 3, 0), (3, 1, 0) \\
120780 & (1, 13, 16470) & (2, 2, 2745), (2, 3, 0), (3, 1, 0) \\
123525 & (1, 13, 13725) & (2, 2, 2745), (2, 3, 0), (3, 1, 0) \\
126270 & (1, 12, 16470) & (1, 13, 13725), (2, 2, 2745), (2, 3, 0), (3, 1, 0) \\
129015 & (2, 2, 2745) & (2, 3, 0), (3, 1, 0) \\
131760 & (2, 2, 2745) & (2, 3, 0), (3, 1, 0) \\
134505 & (1, 13, 16470) & (2, 2, 2745), (2, 3, 0), (3, 1, 0) \\
137250 & (1, 13, 13725) & (2, 2, 2745), (2, 3, 0), (3, 1, 0) \\
139995 & (2, 2, 2745) & (2, 3, 0), (3, 1, 0) \\
142740 & (2, 2, 2745) & (2, 3, 0), (3, 1, 0) \\
145485 & (1, 12, 16470) & (1, 13, 13725), (2, 2, 2745), (2, 3, 0), (3, 1, 0) \\
148230 & (2, 2, 2745) & (2, 3, 0), (3, 1, 0) \\
150975 & (1, 13, 16470) & (2, 2, 2745), (2, 3, 0), (3, 1, 0) \\
153720 & (1, 12, 13725) & (2, 2, 2745), (2, 3, 0), (3, 1, 0) \\
156465 & (1, 13, 13725) & (2, 2, 2745), (2, 3, 0), (3, 1, 0) \\
159210 & (1, 13, 13725) & (2, 2, 2745), (2, 3, 0), (3, 1, 0) \\
161955 & (1, 13, 16470) & (2, 2, 2745), (2, 3, 0), (3, 1, 0) \\
164700 & (1, 13, 13725) & (2, 2, 2745), (2, 3, 0), (3, 1, 0) \\
150975 & (1, 13, 16470) & (2, 2, 2745), (2, 3, 0), (3, 1, 0) \\
153720 & (1, 12, 13725) & (2, 2, 2745), (2, 3, 0), (3, 1, 0) \\
156465 & (1, 13, 13725) & (2, 2, 2745), (2, 3, 0), (3, 1, 0) \\
159210 & (1, 13, 13725) & (2, 2, 2745), (2, 3, 0), (3, 1, 0) \\
161955 & (1, 13, 16470) & (2, 2, 2745), (2, 3, 0), (3, 1, 0) \\
164700 & (1, 13, 13725) & (2, 2, 2745), (2, 3, 0), (3, 1, 0) \\
\hline
\end{longtable}


\subsection{小数点对齐 dcolumn}

\index{宏包!dcolumn}
\index{命令!D{}{}{}}

在 tabular 和 array 环境下定义了列格式选项:
\begin{shaded}
  \begin{Verbatim}
D{输入符号}{输出符号}{小数位数}
\begin{tabular}{|D{.}{,}{2}|D{.}{.}{5}|D{.}{.}{-1}|}\hline
10.33  & 10.33   & 10.33 \\
1000   & 1000    & 1000  \\
5.1    & 5.1     & 5.1   \\
3.14159& 3.14159 & 3.14159 \\\hline
\end{tabular}
可用 \newcolumntype 简化 D 命令
  \end{Verbatim}
\end{shaded}
\textcolor[rgb]{0.50,0.00,1.00}{
注意:
\begin{enumerate}
  \item 如果小数位数不足将溢出。
  \item 小数点对齐会自动转为公式模式,如有文本用$\backslash$text\{\}形式。
\end{enumerate}}
~\\
\begin{tabular}{|D{.}{,}{2}|D{.}{.}{5}|D{.}{.}{-1}|}\hline
10.33  & 10.33   & 10.33 \\
1000   & 1000    & 1000  \\
5.1    & 5.1     & 5.1   \\
3.14159& 3.14159 & 3.14159 \\\hline
\end{tabular}

\begin{shaded}
  \begin{Verbatim}
\usepackage{dcolumn}

%在导言区或表格前加下列命令
\newcolumntype{,}{D{.}{,}{5}}
 %“,”小数点输入,逗号输出,小数位数最多5位
\newcolumntype{d}[1]{D{.}{.}{#1}}
%d{小数位数},输入输出都为.  小数位数自己设定
\newcolumntype{.}{D{.}{.}{-1}}
%“.” 输入输出小数点,小数位数为任意值。

\begin{tabular}{|,|d{5}|.|}\hline
10.33   & 10.33   & 10.33 \\
1000    & 1000    & 1000  \\
5.1     & 5.1     & 5.1   \\
3.14159 & 3.14159 & 3.14159 \\\hline
\end{tabular}
  \end{Verbatim}
\end{shaded}


\begin{tabular}{|,|d{5}|.|}\hline
10.33   & 10.33   & 10.33 \\
1000    & 1000    & 1000  \\
5.1     & 5.1     & 5.1   \\
3.14159 & 3.14159 & 3.14159 \\\hline
\end{tabular}


\subsection{整体缩放表格}


\index{命令!\verb$\scalebox$}

需插图宏包 graphicx,命令如下:
\begin{lstlisting}[language={[LaTeX]TeX}]
      \scalebox{`水平缩放系数`}[`垂直缩放系数`]{`表格`}
\end{lstlisting}

\begin{shaded}
  \begin{Verbatim}
\begin{table}[htbp]
  \centering
  \caption{缩放表格}\label{table_scale}
  \scalebox{1.25}[0.8]{
  \begin{tabular}{lll}
    \toprule
    表格类型 & 所用宏包 & 对应命令\\
    \midrule
    三线表格 & booktabs  & toprule,midrule,bottomrule \\
    斜线表格 & slashbox \\
    彩色表格 & colortbl \\
    自适应表格 & tabularx\\
    合并行列表格 & multirow \\
    长表格& longtable \\
    \bottomrule
  \end{tabular}
  }
\end{table}
  \end{Verbatim}
\end{shaded}

\begin{table}[htbp]
  \centering
  \caption{缩放表格}\label{table_scale}
  \begin{tabular}{lll}
    \toprule
    表格类型 & 所用宏包 & 对应命令\\
    \midrule
    三线表格 & booktabs  & toprule,midrule,bottomrule \\
    斜线表格 & slashbox \\
    彩色表格 & colortbl \\
    自适应表格 & tabularx\\
    合并行列表格 & multirow \\
    长表格& longtable \\
    \bottomrule
  \end{tabular}
\end{table}

\begin{table}[htbp]
  \centering
  \caption{缩放表格}\label{table_scale}
  \scalebox{1.25}[0.8]{
  \begin{tabular}{lll}
    \toprule
    % after \\: \hline or \cline{col1-col2} \cline{col3-col4} ...
    表格类型 & 所用宏包 & 对应命令\\
    \midrule
    三线表格 & booktabs  & toprule,midrule,bottomrule \\
    斜线表格 & slashbox \\
    彩色表格 & colortbl \\
    自适应表格 & tabularx\\
    合并行列表格 & multirow \\
    长表格& longtable \\
    \bottomrule
  \end{tabular}
  }
\end{table}

\subsection{高级表格设置 tabu longtabu}
\index{宏包!tabu}
tabu 宏包可以在表格里实现 verbatim 环境和数学公式,定制表格线型等各种高级功能。
参考 \ref{texref}
\subsection{表格内换行,更改对齐方式,填充序号 makecell}
\index{宏包!makecell}
\index{命令!\verb$\makecell$}
\index{命令!\verb$\thead$}
\index{命令!\verb$\rothead$}

 在正常的表格环境内,每个格的对齐方式是预先定义好的,而且不能用换行符。通过使用 makecell 宏包可以方便地在表格内部进行换行,更改对齐方式,表格内序号填充,斜线表格等操作。

