\section{自定义命令}

\subsection{新建命令 newcommand def}
导言区加入:
\begin{lstlisting}[language={[LaTeX]TeX}]
\newcommand{\fengexian}[1]{
{\centering \makebox[\textwidth][s]{
\textcolor{red}{* * * * \mbox{#1} * * * *}}}}
\end{lstlisting}
定义中 [] 中数字表示变量的个数,在 \{\} 中 \#1 表示变量的位置
\begin{cmd}[label=用法]
\newcommand{\新函数}{新函数定义}
使用时:\新函数
如:\fengexian{我是分割线}
\end{cmd}
\fengexian{我是分割线}

\subsection{修改命令 renewcommand}


\subsection{自定义盒子 newsavebox}
\index{命令!\verb$\newsavebox$}
\index{命令!\verb$\savebox$}
\index{命令!\verb$\usebox$}
\index{命令!\verb$\sbox$}
\index{环境!lrbox}
其中 name 为自定义的名字,用以下代码可以直接用\verb|\usebox{\name}|来节省代码量。排版时 \TeX 只按盒子大小安排空间,若盒子里内容多出盒子空间,会导致与外部内容重叠,反之则出现大段空白。
\begin{lstlisting}
\newsavebox{\name} % 定义盒子名
\savebox( 宽度 , 高度 )[ 位置 ]{\name}{ 内容 } % 定义盒子内容
\sbox{\name}{ 内容 } % \savebox 的简略用法
\begin{lrbox}{\name} % 保存大段文本的盒子形式
    内容
\end{lrbox}
\usebox{\name}  % 引用盒子
\end{lstlisting}
