
\section{文档问题}
\newcounter{buzhou_latex} %不同环境计数器的名称也必需不同
\begin{list}
{\bfseries\sffamily 问题\,\arabic{buzhou_latex}:\hfill}
{\setlength{\parsep}{\parskip}
 \setlength{\itemsep}{0ex plus0.1ex}
 \setlength{\labelwidth}{4em}
 \setlength{\labelsep}{0.2em}
 \setlength{\leftmargin}{6.2em}
 \setlength{\rightmargin}{2em}
 \usecounter{buzhou_latex} \setcounter{buzhou_latex}{0}
 \upshape
}

%\addcontentsline{toc}{subsection}{\qquad$\bigstar$ 问题列表}

\addcontentsline{toc}{subsection}{$\bullet$~合并图表目录}
\item
\color{red}
图表目录合并\\
\normalcolor

用宏包 tocloft




%%%%%%%%%%%%%%%%%%%%%%%%%%%%%%%%%%%%%%%%%%%%%%%%%%%%%%%%%%%%%%

\addcontentsline{toc}{subsection}{$\bullet$~目录链接不正确}

\item
\color{red}
标签页链接不正确\\
\normalcolor

\begin{lstlisting}[language={[LaTeX]TeX}]
\clearpage
\phantomsection
\tableofcontents

\clearpage
\phantomsection
\listoffigures
\listoftables
\end{lstlisting}

%%%%%%%%%%%%%%%%%%%%%%%%%%%%%%%%%%%%%%%%%%%%%%%%%%%%%%%%%%%%%%

\addcontentsline{toc}{subsection}{$\bullet$~表格内换行}
\item
\color{red}
表格内换行,box 内换行\\
\normalcolor

$\backslash$par 命令或\verb|\makecell|命令(需 makecell 宏包)或
\textcolor[rgb]{0.00,0.00,0.63}{使用$\backslash$parbox[t(bsc)默认是 c]\{长度\}\{可使用换行符\}}\\
t(bsc)默认是 c ,可缺省。



%%%%%%%%%%%%%%%%%%%%%%%%%%%%%%%%%%%%%%%%%%%%%%%%%%%%%%%%%%%%%%

\addcontentsline{toc}{subsection}{$\bullet$~PDF 标签乱码}
\item
\color{red}
PDF 标签乱码\\
\normalcolor

重新点 set main file 的图标,重新编译两次\footnote{仅对《 LATEX 入门与提高 第二版》光盘中附赠的 winedt 软件有效,
其他版本的 winedt 在执行完 2 遍 pdflatex 后,再执行 gbk2uni 不带后缀的文件名,再执行一遍 paflatex。}。



%%%%%%%%%%%%%%%%%%%%%%%%%%%%%%%%%%%%%%%%%%%%%%%%%%%%%%%%%%%%%%
\addcontentsline{toc}{subsection}{$\bullet$~图片一直放在文字后}
\item
\color{red}
图片不放在文字前面\\
\normalcolor

 在宏包里使用 flafter 宏包即可


%%%%%%%%%%%%%%%%%%%%%%%%%%%%%%%%%%%%%%%%%%%%%%%%%%%%%%%%%%%%%%

\addcontentsline{toc}{subsection}{$\bullet$~图片强制就地放置}
\item
\color{red}
图片就地放置\\
\normalcolor


        加载 float 宏包,使用\verb$\begin{figure}[H]$命令,\verb$\begin{figure}[h]$中 h 为建议就地放置,H 为命令就地放置。
        H 只能单独使用,不能和其他参数 htbp 混合,否则失去作用。


%%%%%%%%%%%%%%%%%%%%%%%%%%%%%%%%%%%%%%%%%%%%%%%%%%%%%%%%%%%%%%

\addcontentsline{toc}{subsection}{$\bullet$~latex 公式到 word}
\item
\color{red}
公式复制到 word 中\\
\normalcolor


        可先直接复制到 mathtype6.0 ,自动转换成 word 格式


%%%%%%%%%%%%%%%%%%%%%%%%%%%%%%%%%%%%%%%%%%%%%%%%%%%%%%%%%%%%%%

\addcontentsline{toc}{subsection}{字体颜色不均}
\item
\color{red}
用pdflatex生成的字体粗细不一\\
\normalcolor

 字体支持不太好,改用 \LaTeX + dvitopdf 或改用 XeLaTeX (须改设置,utf-8格式,Xecjk 宏包等),


%%%%%%%%%%%%%%%%%%%%%%%%%%%%%%%%%%%%%%%%%%%%%%%%%%%%%%%%%%%%%%


\addcontentsline{toc}{subsection}{$\bullet$~box 环境内的抄录环境}
\item
\color{red}
box 环境内不用用摘录环境\\
\normalcolor

        用 minipage 环境,verbinput 导入外部文件。


%%%%%%%%%%%%%%%%%%%%%%%%%%%%%%%%%%%%%%%%%%%%%%%%%%%%%%%%%%%%%%

\addcontentsline{toc}{subsection}{$\bullet$~PDF 属性中加入作者和文章信息}

\item
\color{red}
在 pdf 属性中加入作者和标题信息。\\
\normalcolor

        使用 hyperref 宏包,在可选项中加入(放在document文档中可以有中文,若在导言区有中文\XeLaTeX 下可能有问题):\\
        \fcolorbox{white}{lightgreen}{
        \parbox{10cm}{
          pdfauthor=\{\},\%作者\\
          pdfkeywords=\{latex\},\%关键词\\
          pdfsubject=\{late beamer asymptote\},\%主题\\
          pdftitle=\{handbook of latex\},\%标题\\
        }}


%%%%%%%%%%%%%%%%%%%%%%%%%%%%%%%%%%%%%%%%%%%%%%%%%%%%%%%%%%%%%%


\addcontentsline{toc}{subsection}{$\bullet$~总页码计数器}
\item
\color{red}
页码后标上总页码对比\\
\normalcolor

使用 lastpage 宏包,用\verb|\pageref{LastPage}|显示总页码
\textcolor[rgb]{1.00,0.00,0.00}{注意此宏包与 movie15 animate 动画影音宏包有冲突。会导致动画和视频不能正常显示}
\begin{shaded}
\begin{Verbatim}
\usepackage{lastpage}
\pageref{LastPage}
\fancyfoot{\thepage/\pageref{LastPage}}
\end{Verbatim}
\end{shaded}
%%%%%%%%%%%%%%%%%%%%%%%%%%%%%%%%%%%%%%%%%%%%%%%%%%%%%%%%%%%%%%

\addcontentsline{toc}{subsection}{$\bullet$~章节页加页眉}
\item
\color{red}
章节页加页眉\\
\normalcolor


在此页加入命令
\begin{lstlisting}[language={[LaTeX]TeX}]
\thispagestyle{fancy}
\fancyhead[R]{\song\wuhao `右页眉内容`}
\fancyhead[L]{\song\wuhao `左页眉内容`}
\end{lstlisting}




%%%%%%%%%%%%%%%%%%%%%%%%%%%%%%%%%%%%%%%%%%%%%%%%%%%%%%%%%%%%%%

\addcontentsline{toc}{subsection}{$\bullet$~目录页加页眉 tocloft}
\item
\color{red}
目录上加页眉 tocloft 宏包,和 titletoc 并用会导致图表目录不能分页\\
\normalcolor

        tocloft 宏包的 在\verb$\tableofcontents$前面加\verb$\tocloftpagestyle{fancy}$,%
        后面加\verb$\thispagestyle{fancy}$默认是 plain,在目录的第一页加上页眉,%
        后\verb$\thispagestyle{fancy}$是在最后一页加页眉。在前面加下述命令,可使目录第一页至最后一页前都加上页眉

\begin{shaded}
\begin{Verbatim}
\makeatletter
\renewcommand{\tocloftpagestyle}[1]
{\def\@cftpagestyle{\pagestyle{#1}}}
\makeatother
\end{Verbatim}
\end{shaded}


%%%%%%%%%%%%%%%%%%%%%%%%%%%%%%%%%%%%%%%%%%%%%%%%%%%%%%%%%%%%%%

\addcontentsline{toc}{subsection}{$\bullet$~目录章节首页加页眉 titletoc}

\item
\color{red}
用 titletoc 宏包,在目录页、章节页加上页眉,可重定义 fancypagestyle\{plain\} 的格式\\
\normalcolor
\begin{shaded}
\begin{Verbatim}
\fancypagestyle{plain}{\pagestyle{fancy}
\end{Verbatim}
\end{shaded}

%%%%%%%%%%%%%%%%%%%%%%%%%%%%%%%%%%%%%%%%%%%%%%%%%%%%%%%%%%%%%%

\addcontentsline{toc}{subsection}{$\bullet$~将文献引用改为上标格式}

\item
\color{red}
引用文献是\verb|\cite{...}|命令,要实现上标有三种方法\\
\normalcolor
\begin{shaded}
\begin{Verbatim}
一、
\newcommand{\upcite}[1]
{\texsuperscript{\textsuperscript{\cite{#1}}}}
二、
\newcommand{\upcite}[1]%
{$^{\mbox{\scriptsize\cite{#1}}}$}
三、
\makeatletter
\def\@cite#1#2%
{\textsuperscript{'{#1\if@tempswa,#2\fi}]}}
\makeatother
\end{Verbatim}
\end{shaded}


%%%%%%%%%%%%%%%%%%%%%%%%%%%%%%%%%%%%%%%%%%%%%%%%%%%%%%%%%%%%%%
\addcontentsline{toc}{subsection}{$\bullet$~附录的节号 section 不对}
\item
\color{red}
附录的 section 节号不对\\
\normalcolor
\begin{shaded}
\begin{Verbatim}
\setcounter{section}{1}
\end{Verbatim}
\end{shaded}

\addcontentsline{toc}{subsection}{$\bullet$~表格编译 label 出错}
\item
\color{red}
编译时表格的 label 后出错\\
\normalcolor

table 表格的 label 后不能加\verb|\\|,longtable 表格和图形环境的
label 可以加。

\addcontentsline{toc}{subsection}{$\bullet$~hyperref 没有超链接}
\item
\color{red}
以前模板的目录没有超链接属性\\
\normalcolor

注意超链接的参数选项设置,有的选项没有设成宏包规定的值则会出错。 pdfborder=\{0 0 0\} 为无边框。 不能写成 no 或 false 。



\addcontentsline{toc}{subsection}{$\bullet$~命令作用范围到了后面的表格}
\item
\color{red}
\verb|\rowcolors| 命令作用范围到了后面的表格\\
\normalcolor

\verb|\rowcolors|必须放在环境中限制其范围,如果外没有表格环境可以在其作用范围前后加上 \{ 和 \} 来实现范围设定。


\addcontentsline{toc}{subsection}{$\bullet$~抄录环境中代码过长}
\item
\color{red}
代码过长,超出了页边范围\\
\normalcolor

1.手动换行

2.用\verb|\footnotesize|等这种原有的 LATEX 字体大小写定义将字体改小,用自定义的 WUHAO 等字体命令会不起作用。
%%%%%%%%%%%%%%%%%%%%%%%%%%%%%%%%%%%%%%%%%%%%%%%%%%%%%%%%%%%%%%


%%%%%%%%%%%%%%%%%%%%%%%%%%%%%%%%%%%%%%%%%%%%%%%%%%%%%%%%%%%%%%
\addcontentsline{toc}{subsection}{$\bullet$~新加宏包后出现no room for ... 的错误}
\item
\color{red}
宏包冲突,出现 no room for ... \\
\normalcolor

 使用 etex 宏包 \verb|\usepackage{etex}|


%%%%%%%%%%%%%%%%%%%%%%%%%%%%%%%%%%%%%%%%%%%%%%%%%%%%%%%%%%%%%%
\addcontentsline{toc}{subsection}{$\bullet$~winedt 局部编译后不自动弹出PDF,且乱码}
\item
\color{red}
winedt7 局部编译后不自动弹出PDF,且乱码,因为编码方式为GBK所致,调用的是zhmCJK的GBK编译宏包,或用 winedt6\\
\normalcolor

 局部编译一整个 xx.tex 的源文件,或者转换文件为 UTF8 的编码方式。


%%%%%%%%%%%%%%%%%%%%%%%%%%%%%%%%%%%%%%%%%%%%%%%%%%%%%%%%%%%%%%
\addcontentsline{toc}{subsection}{$\bullet$~xelatex 编译utf-8格式出错}
\item
\color{red}
原GBK格式另存为UTF-8格式后用xelatex还是会有错误\\
\normalcolor

 新建一个文件,再保存为UTF-8格式就可以了,我也不知道为什么。

\addcontentsline{toc}{subsection}{$\bullet$~no room for a new write}
\item
\color{red}
加入新宏包 tcolorbox 等会出现编译错误 no room for a new write\\
\normalcolor

 加载宏包 \verb|\usepackage{morewrites}|来解决。

%%%%%%%%%%%%%%%%%%%%%%%%%%%%%%%%%%%%%%%%%%%%%%%%%%%%%%%%%%%%%%

\end{list}
