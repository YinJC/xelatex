\bookmarksetup{bold,color=grass}
\chapter{软件安装配置}
\bookmarksetup{bold=false,color={}}
 \thispagestyle{fancy}
\fancyhead[R]{\song\wuhao \leftmark}


\section{安装所需软件}
%
%本文档在 XP 系统下,使用 《LATEX 入门与提高光盘》附赠软件 ChinaTex,和网上下载的宏包,用 winedt 编辑器编辑而成。
%所用到的软件如下,可不用升级的软件,推荐使用其余的软件,会给编辑带来很大方便\upcite{latex}。
%
\begin{table}[htbp]
  \caption{推荐软件}\label{soft_recommend}
  \center
\begin{tabularx}{350pt}{lX}
  \toprule   \rowcolor{backcolor}
  软件名称 & 备注\\
  \midrule
  TEXLIVE 2012 &  tex 编译软件,xelatex,pdflatex等等集成软件包  \\ \rowcolor{lightgray}
  winedt7 &  编辑器\\
  Emacs & 编辑器 \\ \rowcolor{lightgray}
  % &  \\
  Ghostscript 9.05 & EPS 图形相关软件 GSview 软件须装 \\
  Gsview5.0 & EPS 图形查看软件(\textcolor[rgb]{0.00,0.00,1.00}{16417-30959}) \\\rowcolor{lightgray}
  Asymptote & 矢量绘图软件(TEXLIVE中有,但有些有软件错误需从原版修正) \\
  \parbox[t]{6cm}{ImageMagick 6.8.0}& 跨平台的全能图片转换工具\\ \rowcolor{lightgray}
  word2tex2.4.exe   & word 转 TEX,方便进行转换
  注意转之前改成英文名,图片会以文档名带数字的 EPS 形式转换出\\
  GeoGebra & 可视化绘图软件,可以转换成asy和pgf代码。\\ \rowcolor{lightgray}
  Adobe Acrobat 9 Pro & 9 以上的版本用于看 3D 图和其他特效,推荐使用\\
  ccmap,slashbox,shortlst 宏包 & 宏包扩展 \\\rowcolor{lightgray}

  \bottomrule
\end{tabularx}
\end{table}

%
%\noindent
%\fcolorbox{white}{lightgray}{其中 Asymptote 涉及的软件还包括:}
%\begin{enumerate}
%  \item python-2.6.msi
%  \item PIL-1.1.7.win32-py2.6.exe
%  \item animate.sty
%  \item movie15.sty
%  \item animfp.sty
%  \item ocg.sty
%\end{enumerate}

升级TEXLIVE 2012后的所用软件如下:
\section{安装新宏包方法}
将包括新宏包的文件夹放入\verb|D:\texlive\2012\texmf-dist\tex\latex| 文件夹下,进入DOS或终端,运行输入 texhash 命令,mac 和 linux 系统运行 sudo texhash。如果是 TDS 系列的宏包,如 zhmCJK 系列的宏包,放到更上一层的对应的文件夹下,再运行 texhash 命令。
\section{mac 系统安装后设置}
 使用 \XeLaTeX 个人感觉最牛的是终于可以跨平台编译文件,中文字体的问题解决了。对,\XeLaTeX + CTEX 宏包可以跨平台了。经测试,本文档在 mac os 和 winxp、win7 平台上都可编译成功。使用 TEXLIVE2012 在 Win 平台,MacTex 在 Mac os 平台上。Linux 应该也可以,但暂时我没测试过。

mac 下安装完 MacTeX 后还需对 CTEX 的相关字体文件进行更改才能生效。CTeX中默认使用的SimSun等字体在Mac OS中并不存在,取而代之的是“华文宋体”等华文系列的字体。因此如果不配置,会因找不到字体而出现编译错误。另外,在Windows中常用的“隶书”和“幼圆”两种字体,在Mac OS中根本不存在,也没有可以替换的字体。对于缺失的隶书和幼圆,我们使用Windows中的字体,下载地址是
\uline{\url{http://tinker-bot.googlecode.com/files/cfonts.tar.gz}}
我们只用到其中的\fbox{simli.ttf}和\fbox{simyou.ttf}这两个字体。另外的4的Adobe 字体在我的系统中已经自带(也可能是我安装Photoshop时装上的?)
打开应用程序中的“字体册”,点“所有字体”,再在菜单栏中点“文件”-“添加字体”,选择从刚下载的压缩包中解压出的simli.ttf和simyou.ttf,你也可以点击“中文”标签来检查一下四个Adobe字体是否已经自带,如果没有,就在添加字体时把它们也加进来。默认情况下字体会安装到你的用户文件夹下,如果你希望计算机上所有的用户都可以使用这些字体,可以在字体册偏好设置里设置将其安装到系统字体文件夹中(我是这么做的)。当然这需要你有root的口令。
安装好字体后需要修改一下配置文件,在应用程序中打开终端,输入下列命令
\begin{cmd}[label=修改MACTEX字体文件命令,xrightmargin=-0.5cm]
sudo vim /usr/local/texlive/2012/texmf-dist/tex/latex/ctex/fontset/ctex-xecjk-winfonts.def
\end{cmd}
(如果您不会用vim就换个编辑器,比如nano)
输入密码后将其内容替换为:

\begin{lstlisting}[language={[LaTeX]TeX}]
\setCJKmainfont[BoldFont={STHeiti},ItalicFont=STKaiti]
  {STSong}
\setCJKsansfont{STHeiti}
\setCJKmonofont{STFangsong}

\setCJKfamilyfont{zhsong}{STSong}
\setCJKfamilyfont{zhhei}{STHeiti}
\setCJKfamilyfont{zhkai}{STKaiti}
\setCJKfamilyfont{zhfs}{STFangsong}
\setCJKfamilyfont{zhli}{LiSu}
\setCJKfamilyfont{zhyou}{YouYuan}

\newcommand*{\songti}{\CJKfamily{zhsong}} % 宋体
\newcommand*{\heiti}{\CJKfamily{zhhei}}   % 黑体
\newcommand*{\kaishu}{\CJKfamily{zhkai}}  % 楷书
\newcommand*{\fangsong}{\CJKfamily{zhfs}} % 仿宋
\newcommand*{\lishu}{\CJKfamily{zhli}}    % 隶书
\newcommand*{\youyuan}{\CJKfamily{zhyou}} % 幼圆

\endinput
\end{lstlisting}

同理修改另外一个配置文件,命令是
\begin{cmd}[label=修改MACTEX字体文件命令,xrightmargin=-0.8cm]
sudo vim /usr/local/texlive/2012/texmf-dist/tex/latex/ctex/fontset/ctex-xecjk-adobefonts.def
\end{cmd}
加入两行以支持隶书和幼圆,修改后的内容为:
\begin{lstlisting}[language={[LaTeX]TeX}]
\setCJKmainfont[BoldFont=Adobe Heiti Std,ItalicFont=Adobe Kaiti Std]
  {Adobe Song Std}
\setCJKsansfont{Adobe Heiti Std}
\setCJKmonofont{Adobe Fangsong Std}

\setCJKfamilyfont{zhsong}{Adobe Song Std}
\setCJKfamilyfont{zhhei}{Adobe Heiti Std}
\setCJKfamilyfont{zhfs}{Adobe Fangsong Std}
\setCJKfamilyfont{zhkai}{Adobe Kaiti Std}
\setCJKfamilyfont{zhli}{LiSu}
\setCJKfamilyfont{zhyou}{YouYuan}

\newcommand*{\songti}{\CJKfamily{zhsong}} % 宋体
\newcommand*{\heiti}{\CJKfamily{zhhei}}   % 黑体
\newcommand*{\kaishu}{\CJKfamily{zhkai}}  % 楷书
\newcommand*{\fangsong}{\CJKfamily{zhfs}} % 仿宋
\newcommand*{\lishu}{\CJKfamily{zhli}}    % 隶书
\newcommand*{\youyuan}{\CJKfamily{zhyou}} % 幼圆

\endinput
\end{lstlisting}

下面我们来测试配置,使用TeXShop作为IDE,代码存为UTF-8编码,以XeLaTeX编译。
\begin{lstlisting}[language={[LaTeX]TeX}]
\documentclass[UTF8]{ctexart} % 采用Mac字体
%\documentclass[UTF8,adobefonts]{ctexart} % 采用Adobe字体

%以上两行分别测试华文字体和Adobe字体,请交替使用(月下独酌 注)

\title{\LaTeX 中文设置之高层方案}
\author{xiaoyong}
\date{\today}

\begin{document}
\maketitle

\begin{center}
  1. 字体示例:\\
  \begin{tabular}{c|c}
    \hline
    \textbf{\TeX 命令} & \textbf{效果}\\
    \hline
    \verb|{\songti 宋体}| & {\songti 宋体}\\
    \hline
    \verb|{\heiti 黑体}| & {\heiti 黑体}\\
    \hline
    \verb|{\fangsong 仿宋}| & {\fangsong 仿宋}\\
    \hline
    \verb|{\kaishu 楷书}| & {\kaishu 楷书}\\
    \hline
    \verb|{\lishu 隶书}| & {\lishu 隶书}\\
    \hline
    \verb|{\youyuan 幼圆}| & {\youyuan 幼圆}\\
    \hline
  \end{tabular}
\end{center}

\begin{center}
  2. 字号示例:\\
  {\zihao{0}初号}
  {\zihao{1}一号}
  {\zihao{2}二号}
  {\zihao{3}三号}
  {\zihao{4}四号}
  {\zihao{5}五号}
  {\zihao{6}六号}
  {\zihao{7}七号}
  {\zihao{8}八号}
\end{center}

\end{document}
\end{lstlisting}

\section{winedt使用技巧}
\subsection{编译方法}
本文档的编译方法,安装完各宏包后,按以下几步,有宏包冲突可先按 Enter 忽略错误强行编译。

\begin{enumerate}
  \item \XeLaTeX (或pdflatex) XELATEX.TEX
  \item Accessories$\rightarrow$ make index 生成索引
  \item \XeLaTeX (或pdflatex) XELATEX.TEX
  \item \XeLaTeX (或pdflatex) XELATEX.TEX 加入索引目录
\end{enumerate}


本文档编译时会出现 dingbat.sty 和 marvosym.sty 错误,是因为引用了太多的符号宏包,其中符号宏包定义的符号相冲突,把对应冲突的符号(对应 dingbat 里的 \fbox{checkmark}和 marvosym 中的\fbox{Letter})注释掉,再 texhash 一下,即可。具体操作见\ref{sym_clash}。


PDFLaTeX 可以直接运行 PDFTeXify 实现以上 4 个步骤,\XeLaTeX 还需额外配置一下。

注:以上都必须将主工程目录设置一下。在左上角的 + 号符号,\fbox{SET MAINFILE}。

编译方法2: windows 下可以直接建立 BAT 文件来设置这些命令顺序执行,双击 BAT 文件即可。本文为 MAKEPDF.BAT 文件,内容如下所示:
\lstinputlisting{MAKEPDF.BAT}
\subsection{winedt 6, 7 设置快捷键}
\textcolor[rgb]{1.00,0.00,0.00}{winedt5.6 快捷键设置:}

options $\rightarrow$ menu setup $\rightarrow$ 双击对应选项添加 shortcut。
推荐设置添加环境列表、表格、公式、编译的快捷方式。

CTRL+ALT+$\rightarrow$:对选中的对象每行添加字符,可加 \% 以注释内容,反方向箭头为去除。

WINEDT 6 采用自己配置的方法设置快捷方式,使用
options$\rightarrow$menus and toolbar$\rightarrow$main menu 里,在对应的选项后加上 SHORTCUT="快捷键组合" 即可。


WINEDT 7 同上,使用
options$\rightarrow$options Intereface$\rightarrow$menus and toolbar$\rightarrow$main menu 里,在对应的选项后加上 SHORTCUT="快捷键组合" 即可。如下所示,将 Itemize 的快捷键设为“Ctrl+Alt+1”,需在对应代码后加入 SHORTCUT="Ctrl+Alt+1"。

\begin{lstlisting}
      ITEM="Itemize"
      CAPTION="&Itemize"
      IMAGE="ListItem"
      MACRO="LetReg(9,'List Items');LetReg(8,'itemize');"+
            "Exe('%b\Menus\Insert\List.edt');"
      REQ_DOCUMENT=1
      SHORTCUT="Ctrl+Alt+1"
\end{lstlisting}

然后保存,右键 Load script 或 F9 加载一下即可,有的快捷键已占用会出错,注意设置键。

\subsection{winedt 6 自动加载 gbk2uni}

在 WinEdt 6 或 WINEDT7 的安装文件夹下找到\verb|\exec\execompiler.edt| 文件,用记事本打开,在文本内容中的 END; 前添加如下代码:
\scriptsize

\begin{cmd}[label= 复制 winedt 代码]
IfStr("%!9","LaTeX", "=",!"JMP(!'gbk2uni');");
IfStr("%!9","PDFLaTeX", "=",!"JMP(!'gbk2uni');");
IfStr("%!9","TeXify", "=",!"JMP(!'gbk2uni');");
IfStr("%!9","PDFTeXify","=",!"JMP(!'gbk2uni');");
JMP(!'gbk2uni-Done');

:gbk2uni:: ================================================

IfFileExists("%!6\%N.out","",!"JMP(!'gbk2uni-Done');");
IfStr("%$('%!9-WinEdt_Console');",'1','=',>
!|RunConsole('gbk2uni.exe "%N"','%!6','%!9 ...',1,1);|,>
!|WinExe('','gbk2uni.exe "%N"','%!6','%!9 ...', %!0, %!2,>
'', '%b\_Out.log', '%b\_Err.log',%!1);|);

:gbk2uni-Done:: ================================================
\end{cmd}


\normalsize
附件中的 TXT 文档中复制粘贴,见\textattachfile{winedt6_gbk2uni.txt}{\textcolor[rgb]{0.00,0.00,1.00}{代码粘贴附件}}
\subsection{注册提示框}

winedt5.6:
\begin{shaded}
\noindent
user: Hard Wisdom\\
code: 1135362106278309830
\end{shaded}

winedt6:
\begin{shaded}
options菜单$\rightarrow$options$\rightarrow$ advanced configuration$\rightarrow$Event Handlers$\rightarrow$Exit\\
在 END; 前加一行\\
\wuhao\verb| RegDeleteValue('HKEY_CURRENT_USER', 'Software\WinEdt 6', 'Inst');|\\
\normalsize 保存后在 Exit 点鼠标右键 Execute Script
\end{shaded}

winedt7:
\begin{shaded}
开始$\rightarrow$运行$\rightarrow$regedit$\rightarrow$HKEY CURRENT USER$\rightarrow$SOFTWARE\\
删除其中的Winedt 和 Winedt7 两个文件夹一样的东西,整体删除。
\end{shaded}

\subsection{winedt 打开 utf-8 文件}
方案一:是在文件的开头加上一行
\begin{cmd}
% !Mode:: "TeX:UTF-8"
\end{cmd}
这样这个tex文档就会被默认用UTF-8的格式打开。
方案二:是在options→ configuration wizard→wrapping 下 UFT-8 format中加上Tex,改成
\begin{cmd}
Tex;UTF-8|ACP;EDT;INI
\end{cmd}

就可以了。这样所有的tex文档都被默认是utf-8文档打开。
但是winedt还没有办法很好支持两种以上的utf-8文档,比如同时输入中文和韩语。

注意:如上修改完毕后,请关闭当前文件,然后打开即可显示正常。


%\section{pgf 安装配置}
%\begin{enumerate}
%  \item 下载最新的 pgf 宏包
%  \item 删除旧宏包(否则会出错),删除以下文件夹
%  \begin{lstlisting}
%1.c:\ChinaTeX\texmf\doc\generic\pgf
%2.c:\ChinaTeX\texmf\tex\context\third
%3.c:\ChinaTeX\texmf\tex\latex\pgf
%4.c:\ChinaTeX\texmf\tex\plain\pgf
%5.c:\ChinaTeX\texmf\tex\generic\pgf
%  \end{lstlisting}
%  \item 将新宏包解压放到\verb|c:\ChinaTeX\texmf|内,更新目录,进入 DOS 执行 texhash 命令
%\end{enumerate}

\section{ASY 安装配置}
安装完相应的软件后在 \verb|C:\Documents and Settings\Administrator\.asy|中添加 \textattachfile{config.asy}{\uline{\textcolor[rgb]{0.00,0.00,1.00}{CONFIG.ASY}}} 文件,内容如下:
\begin{cmd}
import settings;
gs="D:\Program Files\gs\gs9.00\bin\gswin32c.exe";
psviewer="D:\Program Files\Ghostgum\gsview\gsview32.exe";
pdfviewer="D:\Program Files\Adobe\Acrobat 9.0\Acrobat\Acrobat.exe";
python="D:\Python26\python.exe";
\end{cmd}

