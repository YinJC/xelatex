\section{盒子设计}
\subsection{盒子命令集}

如表\ref{box_command}所示:\\

\begin{table}[htbp]
\centering
\caption{盒子对应命令} \label{box_command}
\rowcolors{2}{lightgray}{white}

\begin{tabularx}{14cm}{lXX}
  \toprule
  盒子种类 & 对应命令 & 参数含义 \\
  \midrule
  无框盒子 & \verb|\mbox{文本}| &   \\
  可调参数无框盒子 & \verb|\makebox[宽度][位置]{文本}| & 先宽度后位置, l,r,s,缺省(c) 左、中、右、撑满四种位置\\
  有框盒子 & \verb|\fbox{文本}| &   \\
  可调参数有框盒子 &\verb|\fbox[宽度][位置]{文本}| &  \\
  盒子复用 & \parbox{6cm}{\verb|\newsavebox{盒子名}|\\\verb|\sbox{盒子名}{文本}|\\\verb|\usebox{盒子名}|}
    & 新建盒子名,以新建盒子名保存盒子,复用盒子  \\
  多行文本盒子 & \verb|\parbox[位置]{宽度}{文本}| &   \\
% 可抄录多行文本盒子 &  &  \\
%   &  &  \\
%   &  &  \\
  \bottomrule
\end{tabularx}
\end{table}


\subsection{多行文本,parbox,minipage}
\index{命令!\verb$\begin{minipage}$}
\index{命令!\verb$\parbox$}

\begin{shaded}
  \begin{Verbatim}
    \parbox[位置]{宽度}{内容}
    位置有 b t 两种,盒子内文本基线与外面基线平齐
    \begin{minipage}[位置]{宽度}
        ...
    \end{minipage}
  \end{Verbatim}
\end{shaded}
minipage 盒子里可使用抄录环境 verbatim,parbox 不行。用法如下
\begin{enumerate}
  \item 盒子,表格里的多行内容添加,
  \item 图文混排,图形旁边多行文本的添加,两个子图,表格并排放置
\end{enumerate}

\subsection{外框,背景色,framed,shade}
\index{命令!\verb$\begin{shaded}$}
\index{命令!\verb$\begin{framed}$}
\index{宏包!framed}

\begin{lstlisting}[language={[LaTeX]TeX}]
\usepackage{framed}
\definecolor{shadecolor}{rgb}{0.92,0.92,0.92}
\begin{document}
\begin{shaded}
`文本背景色效果`
`支持跨页,多行。`
\begin{equation}
  \int_a^b f(x) dx
\end{equation}
\end{shaded}
\begin{framed}
`文本框效果`
\begin{equation}
  \int_a^b f(x) dx
\end{equation}
\end{framed}
\end{document}
\end{lstlisting}

对应效果如下\textcolor[rgb]{1.00,0.00,0.00}{\footnote{注意,framed 宏包与 fancybox 宏包冲突}}:\\
文本背景色效果
支持跨页,多行。
\begin{equation}
  \int_a^b f(x) dx
\end{equation}

\begin{shaded}
文本背景色效果\\
支持跨页,多行。
\begin{equation}
  \int_a^b f(x) dx
\end{equation}
\end{shaded}

\begin{framed}
文本框效果\\
支持跨页\\
多行文本公式\\
框线厚可调\\
\begin{equation}
  \int_a^b f(x) dx
\end{equation}
\end{framed}

%\subsection{圆角阴影盒子 fancybox}
%\index{命令!\verb$\ovalbox$}
%\index{命令!\verb$\Ovalbox$}
%\index{命令!\verb$\shadowbox$}
%\index{命令!\verb$\doublebox$}
%\index{宏包!fancybox}
%
%
%注意与 framed 宏包冲突,代码:
%\begin{shaded}
%  \begin{Verbatim}
%    \ovalbox{圆角盒子}
%    \Ovalbox{粗圆角盒子}
%    \shadowbox{阴影角盒子}
%    \doublebox{双边框盒子}
%  \end{Verbatim}
%\end{shaded}
%效果如下所示:\\
%    \ovalbox{圆角盒子}
%    \Ovalbox{粗圆角盒子}
%    \shadowbox{阴影角盒子}
%    \doublebox{双边框盒子}



\subsection{彩色圆框,tcolorbox,mdframed}

\subsubsection{Colored boxes}

\begin{tcolorbox}[colback=red!5,colframe=red!75!black]
  My box.
\end{tcolorbox}

\begin{tcolorbox}[skin=bicolor,colback=blue!5,colframe=blue!75!black,title=My title]
  My box with my title.
\end{tcolorbox}

\begin{tcolorbox}[colback=green!5,colframe=green!75!black]
  Upper part of my box.
  \tcblower
  Lower part of my box.
\end{tcolorbox}

\begin{tcolorbox}[skin=bicolor,colback=yellow!5,colframe=yellow!75!black,title=My title]
  I can do this also with a title.
  \tcblower
  Lower part of my box.
\end{tcolorbox}

\begin{tcolorbox}[colback=yellow!10,colframe=red!75!black,lowerbox=invisible,
  savelowerto=\jobname_ex.tex]
  Now, we play hide and seek. Where is the lower part?
  \tcblower
  I'm invisible until you find me.
\end{tcolorbox}

\begin{tcolorbox}[skin=bicolor,colback=yellow!10,colframe=red!75!black,title=Here I am]
  \input{\jobname_ex.tex}
\end{tcolorbox}


\begin{tcolorbox}[colback=blue!50,colframe=blue!25!black,coltext=yellow,
    fontupper=\Large\bfseries,arc=6mm,boxrule=2mm,boxsep=5mm]
  Funny settings.
\end{tcolorbox}


\subsubsection{\LaTeX-Examples}

\begin{tcblisting}{colback=red!5,colframe=red!75!black}
This is a \LaTeX\ example:
$\displaystyle\sum\limits_{i=1}^n i = \frac{n(n+1)}{2}$.
\end{tcblisting}


\subsubsection{Theorems}

\newcounter{mytheorem}[section]
\def\themytheorem{\thesection.\arabic{mytheorem}}

\tcbmaketheorem{theo}{Theorem}{skin=bicolor,fonttitle=\bfseries\upshape, fontupper=\slshape,
     arc=0mm, colback=blue!5,colframe=blue!75!black}{mytheorem}{theorem}

\begin{theo}{Summation of Numbers}{summation}
  For all natural number $n$ it holds:\\[2mm]
  $\displaystyle\sum\limits_{i=1}^n i = \frac{n(n+1)}{2}$.
\end{theo}

We have given Theorem \ref{theorem:summation} on page \pageref{theorem:summation}.
\subsection{旋转任意对象}
\index{命令!\verb$\rotatebox$}
\begin{shaded}
  \begin{Verbatim}
    \rotatebox[origin=,x=,y=,units=]{角度}{对象}
\begin{center}
\rotatebox{90}{
    \begin{tabular}{|l|l|}
      \hline
      % after \\: \hline or \cline{col1-col2} \cline{col3-col4} ...
      1 & 旋转表格 \\\hline
      表格 & 旋转 \\\hline
      \hline
    \end{tabular}}
\end{center}
  \end{Verbatim}
\end{shaded}

参数含义如下:\\
\begin{description}
  \item[origin :]旋转基准点;
  \item[x,y :]旋转点到基准点的位移;
  \item[units :] 旋转角度单位,units=-360 表示为顺时针,单位为度;units=6.283185,表示逆时针,单位为弧度。
\end{description}
\begin{center}
\rotatebox{90}{
    \begin{tabular}{|l|l|}
      \hline
      % after \\: \hline or \cline{col1-col2} \cline{col3-col4} ...
      1 & 旋转表格 \\\hline
      表格 & 旋转 \\\hline
        \end{tabular}}
\end{center}

\subsection{缩放任意对象}
\index{命令!\verb$\scalebox$}
\index{命令!\verb$\width$}
\index{命令!\verb$\height$}
\index{命令!\verb$\totalheight$}
\index{命令!\verb$\depth$}
\begin{itemize}
  \item 尺寸测量命令有\verb$\width$,\verb$\height$,\verb$\totalheight$,\verb$\depth$。
  \item 如果高度或宽度用 ! 号来表示,对象保持原高度不变进行缩放。
\end{itemize}


\begin{shaded}
  \begin{Verbatim}
    \scalebox{水平缩放系数}[垂直缩放系数]{对象}
    \resizebox{宽度}{高度}{对象}
    \resizebox*{宽度}{总高度}{对象}
\begin{center}
\scalebox{5}[1]{缩} \scalebox{2}[4]{放}\\
\resizebox{5\width}{\height}{缩}
\resizebox{2\width}{4\height}{放}
\end{center}
  \end{Verbatim}
\end{shaded}
效果如下:
\begin{shaded}
\begin{center}
 \scalebox{5}[1]{缩} \scalebox{2}[4]{放}\\
\resizebox{5\width}{\height}{缩}
\resizebox{2\width}{4\height}{放}
\end{center}
\end{shaded}

\subsection{镜像任意对象}
\index{命令!\verb$\reflectbox$}

\begin{shaded}
  \begin{Verbatim}
    \reflectbox{对象}
    相当于
    \scalebox{-1}{1}{对象}
    \begin{center}
      镜像 \reflectbox{\color[gray]{0.6}{镜像}}
    \end{center}
  \end{Verbatim}
\end{shaded}

效果如下:
\begin{shaded}
    \begin{center}
      镜像 \reflectbox{\color[gray]{0.6}{镜像}}
    \end{center}
\end{shaded}


