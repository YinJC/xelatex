\chapter{摘~~~~要}
 \thispagestyle{fancy} \fancyhead[R]{\song\wuhao 摘要}
本文档是自己在 \LaTeX 学习过程中为节省时间而做的学习手记,比较方便有基础的人进行命令查询和扩展功能 DIY。初学入门者可能会对其中省略的一些步骤和基础知识感到困感。

在学习 latex 的过程中,发现了很多新的简便用法,虽然出来的效果一致,但方法简便了许多,这个可以从源文件的代码中看出来,后面写的代码比以前简洁许多,但编译出来的效果是一样的。

有热心网友村长 laomaicunzhang@sina.com 提出相关修改意见,在此表示感谢。

学习 emacs ,越来越发现其强大,而且里面的 org-mode 模式可以发布成网页,PDF,BEAMER,TEX 等各种格式,极大地提高了自己的效率,但是其对 PDFLaTeX 中文支持不太好,所以升级到 \XeLaTeX ,插入图形同时支持(eps,jpg,png,pdf),很强大,因为一些矢量图作图软件(asy,graphviz)只有EPS 格式是矢量的。

~\\ \color{info} \indent
偶然的机会遇到\href{http://www.onefoundation.cn/}{\textcolor[rgb]{0.00,0.00,1.00}{壹基金}},觉得这是一件很伟大的公益事业。能最大限度地发动和鼓励更多的人为这个世界奉献自己的爱心,
在这里借这个机会让更多的人知道。发送 1 到 1069999309\footnote{\textcolor[rgb]{1.00,0.50,0.50}{会替你从手机中捐出 1 元到基金会,用于赈灾和其他公益活动}}。\\


 \heartpar{ 虽然不能彻底改变整个世界,
但我们可以尽自己的一份力让这个世界变得比以前更好一点} \color{black}

%
%~\\[2cm]
%\begin{flushright}
%\begin{minipage}{8cm}
%\begin{asy}
%texpreamble("\usepackage{CJK}
% \AtBeginDocument{\begin{CJK*}{GBK}{kai}}
%   \AtEndDocument{\clearpage\end{CJK*}}");
%import three;
%size(200);
%currentprojection=orthographic(-2,-2,1.5);
%path[]g=texpath("LATEX 笔记");
%for(path p:g){
%draw(path3(p),red+1pt);
%draw(extrude(p,2Z),yellow);
%draw(shift(2Z)*path3(p),red+1pt);
%}
%\end{asy}
%\end{minipage}
%\end{flushright}
